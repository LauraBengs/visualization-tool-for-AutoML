% !TEX root = ../main.tex
%
\chapter{Related Work}
\label{sec:related}

In this chapter related work is being discussed. Firstly an overview of existing visualisation tools for AutoML is given, depicting the advantages and drawbacks of those tools. Subsequently the presented tools are compared and weak points are gathered in order to find starting points for improvement. To wrap things up, the chapter ends with a conclusion.

\section{Overview of existing visualisation tools for AutoML}
\label{sec:realted:overview}

As there exist several different visualisation tools for AutoML systems and some research has been done in this area, in the following an overview is given.
\\ \\
ATMSeer \cite{atmseer} was the first of its kind visualisation system \cite{trustautoml} and its target user group are machine learning experts \cite{atmseer}. It is a real-time application that provides transparency as well as controllability for ATM\footnote{ATM is a AutoML system \cite{xautoml}.} \cite{xautoml}. The tool offers multi-granularity visualisation of model selection as well as hyperparameter tuning, enabling the user to monitor the process, analyse searched models and adjust the search space \cite{aiviz, trustautoml, atmseer, hypertendril}. Furthermore ATMSeer offers a visual summary of the search models and is not algorithm specific but can support analysing machine learning models that have been generated with various algorithms \cite{atmseer}. On the downside the tool is unable to present an overview of the AutoML process and the created model pipelines \cite{trustautoml}. In addition it is only possible to analyse the last step and the analysis of pipelines with a more complex structure\footnote{A pipeline with a more complex structure is a directed acyclic graph for example \cite{pipelineprof}.} as well as longer pipelines are not supported \cite{pipelineprof}.
\\ \\
AutoAiViz \cite{aiviz} is a post-hoc visualisation system for SMAC\footnote{SMAC stands for Sequential Model-based Algorithm Configuration and can be seen as an AutoML system \cite{smac}.} that uses conditional parallel coordinates for visualisation \cite{xautoml, hypertendril}. This allows users to control the amount of visible information detail and to regard the pipelines and hyperparameters of each step individually \cite{aiviz, pipelineprof}. Unfortunately AutoAiVis can not be used autonomous but is integrated in an existing AutoML system \cite{aiviz}. Furthermore it requires a fixed pipeline structure \cite{xautoml} and is restricted to linear models, therefore users are unable to explore pipelines with a more complex structure \cite{pipelineprof}.
\\ \\
CAVE\footnote{CAVE stands for Configuration Assessment, Visualisation and Evaluation \cite{cave}.} \cite{cave} is a visualisation tool for SMAC that enables user to analyse the generated machine learning pipeline by generating reports and figures based on all available empirical data \cite{xautoml, cave}. The drawbacks of CAVE include that it requires a fixed pipeline structure and evaluation is only possible post-hoc \cite{xautoml}. Building upon this DeepCAVE \cite{deepcave} was developed. In contrast to CAVE, DeepCAVE supports the evaluation already during calculations. It is an interactive, optimizer agnostic\footnote{Optimiser agnostic means that the tool enables multi-fidelity and multi-objective optimisation.} framework, that makes it possible to analyse and monitor optimisation procedures for AutoDL\footnote{AutoML for deep learning is also referred to as AutoDL \cite{deepcave}.} systems\cite{deepcave}.
\\ \\
Hypertendril \cite{hypertendril} is a web-based visual analytics systems for hyperparameter optimization (HPO) search strategies that additionally provides estimates of hyperparameter importance \cite{hypertendril, xautoml}. Therefore a parallel coordinate view is combined with a scatter plot of sampled values of a limited set of hyperparameters over time \cite{xautoml}.
\\ \\
HyperTuner \cite{hypertuner} enables user to analyse the hyperparameter search by generating interactive visual analytics of the impact of each hyperparameter on the model performance \cite{hypertuner, hypertendril}. Therefore multiple scatter plots are created, visualising a batch of experiments \cite{hypertendril}.
\\ \\
Optuna \cite{optuna} is an open-source optimisation software which can be used to visualise evaluated hyperparameter values, relationships between hyperparameters or feature importance and allows users a real time visualisation and analysis of studies through a web-dashboard \cite{xautoml, optuna}. 
\\ \\
Pipeline Profiler \cite{pipelineprof} enables the exploration and visual comparison of machine learning pipelines that have a more complex structure by creating a single directed acylic graph \cite{xautoml, pipelineprof}. Through that multiple AutoML systems can be evaluated and compared \cite{pipelineprof}. The disadvantages include that Pipeline Profiler is not best suitable for numerical parameters and that it is difficult to compare pipelines with different structures, as individual graphs could be hard to see on the Pipeline Comparison View \cite{pipelineprof}.
\\ \\
Remap \cite{remap} is a tool for the search of neural architectures. It provides users with a visual overview of baseline models and allows visual comparison of those models which leads to a better understanding of the different underlying architectures \cite{remap}. Major drawbacks of this tool are that it can only show linear pipelines and that Remap was developed to visualise neural network architectures which leads to the system not being suitable for different approaches of learning \cite{pipelineprof}.
\\ \\
VisualHyperTuner \cite{visualhypertuner} is a web-based system that visualises the relationship between hyperparameters and performance but is focused on the hyperparameter tuning process rather than the visualisation \cite{visualhypertuner, aiviz}.
\\ \\
XAutoML \cite{xautoml} is a visualisation tool that is independent of the used AutoML system. It allows users to gain further insights by comparing pipeline candidates, analysing the optimisation procedure and inspecting single models as well as machine learning ensembles. Users criticized an information overflow and wished for an option to view less information at once for example by using additional layers. Another major drawback of XAutoML is that results are only available post-hoc and the user can not participate during the optimisation run \cite{xautoml}. Furthermore it is not optimiser agnostic and focuses on the analysis for classification tasks \cite{deepcave}.
\\ \\
For more details about the different visualisation tools, the cited literature stated after the names is recommended. Further visualisation tools for AutoML that have not been discussed in detail here include NNI, AutoWeka, IOHanalyzer, Google Vizier, Visus, TwoRavens, Snowcat, RegressionExplorer, ModelTracker, Squares, TreePOD, Beames, EnsembleMatrix as well as SigOpt.

\section{Comparsion and weak points of existing visualisation tools}
\label{sec:realted:comp}

In order to gain a better overview of the introduced visualisation systems for AutoML a comparison can be seen in table \ref{tab:comparsion}. Thereby the tools are compared regarding their form of the analysis, whether the visualisation took place during calculations (ad-hoc) or was made available afterwards (post-hoc). It is considered if the visualisation tool supports complex structured pipelines or not and whether it is autonomous, i.e. the visualisation tool is not integrated into an existing AutoML systems but transferable and can be used with multiple different systems. Furthermore it is checked whether the tool is optimiser agnostic and a brief description of how the visualisation of the AutoML process is realized is given.
\begin{table}[h!]
    \centering
    \resizebox{\columnwidth}{!}{
    \begin{tabular}{|c|c|>{\centering\arraybackslash}p{2cm}|c|>{\centering\arraybackslash}p{2cm}|>{\centering\arraybackslash}p{3cm}|}
         \hline 
        AutoML system & analysis & complex structured pipelines & autonomous & optimiser agnostic & visualisation\\ \hline
        ATMSeer & ad-hoc & \ding{55} & \ding{51} & & conditional parallel coordinates\\ \hline
        AutoAiViz & post-hoc & \ding{55} &  \ding{55} &  &\\ \hline
        CAVE & post-hoc &  & \ding{55} & \ding{55} & \\ \hline
        DeepCAVE & ad-hoc &    &  & \ding{51} &\\ \hline
        Hypertendril & & & \ding{55} & & parallel coordinate view combined with a scatter plot\\ \hline
        HyperTuner & & & & & multiple scatter plots\\ \hline
        Optuna & ad-hoc & & \ding{55} &  &\\ \hline
        Pipeline Profiler & & & \ding{55} && single directed acylic graph\\ \hline
        Remap & & \ding{55} &  &  & \\ \hline
        VisualHyperTuner & & &  & & \\ \hline
        XAutoML & post-hoc & & \ding{51} & \ding{55}  &\\ \hline   
    \end{tabular}}
    \caption{Comparsion of different visualisation tools for AutoML systems}
    \label{tab:comparsion}
\end{table}

As there are multiple individual tools with different benefits and weak points each, it can be difficult to keep in mind where there is potential for improvement. In the following the disadvantages of existing visualisation tools for AutoML are revised. 
\begin{enumerate}
    \item Visualisation tools are recurrently unable to give an overview of the whole AutoML process, making it impossible for the user to fully understand what is going on.
    \item Visualisation tools often offer a large amount of information at once, so that users have difficulties recognising important information and are overwhelmed more quickly. 
    \item Most visualisation tools only offer a post hoc analysis of the AutoML process. As a result users are unable to interact with the system during the calculations and the AutoML systems remains its black box character.
    \item Visualisation tools are often integrated into existing AutoML systems meaning they are not transferable and can not be used autonomous.
    \item Most visualisation systems are not optimiser agnostic, meaning they do not have have multi objective and multi fidelity in mind.
    \item Most visualisation tools require a fixed pipeline structure in advance as well as they are restricted to linear pipelines, therefore being unable to visualise complex structures.
\end{enumerate}

\section{Conclusion}
\label{sec:related:conclusion}

In this chapter an overview of existing visualisation tools for AutoML was given and related work was discussed. Although there are already multiple approaches on how to increase interpretability and understandability of AutoML systems through visualisation, there is still room for improvement. The previously described disadvantages of existing visualisation tools might present a good starting point for developing an improved tool. Such tool should be able to given an overview of the whole AutoML process and should simultaneously be interactive in order to allow the user to choose what and how much information is being displayed at a time. Furthermore an improved visualisation tools should allow an ad-hoc analysis, enabling the user to follow along the AutoML process already during calculations. Additionally it should be autonomous and not be integrated into an existing AutoML systems as well as it should be optimiser agnostic. Last but not least it is worthwhile to develop a tool that also supports more complex structured pipelines and is not restricted to linear pipelines.